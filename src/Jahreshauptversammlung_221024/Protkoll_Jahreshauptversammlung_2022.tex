%!TEX TS-program = xelatex
%!TEX encoding = UTF-8 Unicode

\documentclass{scrartcl}

\usepackage[]{geometry}
\usepackage{fontspec}
\usepackage{xcolor}


\setmainfont{DINNextLTPro}[ 
	Path = ../../template/,
	Extension = .otf,
	BoldFont = *-Bold,
	ItalicFont= *-Italic,
	BoldItalicFont= *-BoldItalic,
	UprightFont = *-Regular
	]


\geometry{top =50mm}
\usepackage{enumitem}


\setlist[description]{
	font={\bfseries}
}


\definecolor{footercolor}{HTML}{595959}

\usepackage[]{scrlayer-scrpage}
\clearpairofpagestyles
\chead{\includegraphics[width=0.6\columnwidth]{../../template/logo.pdf}}
\ofoot{\pagemark}
\cfoot{\small\color{footercolor}
Träger- und Förderverein MINT Oberland | Bruggfeldstraße 5 | 6500 Landeck \par
office@mint.tirol | www.mint.tirol | ZVR-1223647736\par
AT09 3699 0000 0913 4099 | RBRTAT22
}

\usepackage[utf8]{inputenc}
\usepackage{graphicx}

\usepackage[german]{babel}
\usepackage{minutes}
\usepackage{caption}
\usepackage{biblatex}


\newcommand\vereinName{Träger und Förderverein Mint Oberland}
\newcommand\address{Bruggfeldstraße 5, 6500 Landeck}
\newcommand\eMail{office@mint.tirol}
\newcommand\obmann{Philipp Machac}





\minutesstyle{header={list},columns={1},contents={true}}


% ----------------------------------------------------------------

\begin{document}




\begin{Minutes}{Protokoll der Jahreshauptversammlung vom 24.10.2022}

  \subtitle{\vereinName}
  \moderation{\obmann}
  \minutetaker{Simon Abler}
  \participant{
    \nohyphenation {
      Philipp Machac,
      Rainer Haag,
      Simon Abler,
      Frank Bilger,
      Nikolas Bilger,
      Thurner Tanja,
      Reinhard Machac,
      Werner Starjakob,
      Marco Handle,
      Thomas Zangerl,
      Reinhold Mungenast,
      Schröder Klaus (16:28)}
  }

  \minutesdate{24 Oktober 2022}
  \starttime{16:03}
  \endtime{16:42}
  \location{Lantech Besprechungsraum Klein}
  %\missing{}
  \maketitle
  \newpage

  % Header end ---------------------------------------------------------
  \topic{Begrüßung durch den Vorsitzenden des Vorstands}
  Begrüßung durch Obmann \obmann.
  \subtopic{Feststellung der Beschlussfähigkeit}
  Beschlussfähigkeit wurde festgestellt.

  %---------------------------------------------------------

  \topic{Ansprache des Obmannes}

  Dank an die Unterstützer und Sponsoren, Anmerkung: Mitgliedsbeiträge werden aktuell nicht fristgerecht eingezahlt.
  Kurze Vorstellung des Vereins MINT:
  \begin{itemize}
    \item Ziele: Kinder und Jugendliche fördern
    \item Abwanderung verhindern
    \item Unsere Region stärken
  \end{itemize}
  Geplante Maßnahmen:
  \begin{description}
    \item [Elementarpädagogik:] Brückenbauen, Vorschläge
    \item [Volksschule:] wird schon interessanter für unsere Ziele, Vorträge und Exkursionen Interesse wecken
    \item [SEK1:] Realistisches Berufsbild vermitteln, zB. über Workshops im MINT Lab, Vorträge duch neutralen Partner MINT an der Schule
    \item [SEK2:] Region sollte besser präsentiert werden, Stichwort: \glqq Auch bei uns gibt es attraktive Arbeitsgeber\grqq{}
    \item [Bevölkerung:] Bewusstseinsschaffung in der Bevölkerung
  \end{description}

  %---------------------------------------------------------

  \topic{Projektvorstellungen}

  \subtopic{MINT-Woche 2021 und 2022}
  Vorstellung des Projektes durch \obmann.\\

  \textbf{Bereich SEK1:}\\
  In den Berufsorientierung Pflichtwoche der NMS möchte MINT den Kindern auch einen Einblick in die Berufswelt geben. Statt eine Woche in der gleichen Firma, wird die Woche in 4 Tage geteilt und die Jugendlichen rotieren durch verschiedene Firmen.
  Es wird versucht die MINT-Woche im Lantech durchzuführen, um einen fixen Standort zu vermitteln, falls dies nicht funktioniert, organisieren wir den Transport vom Lantech. Ziel ist es den Kindern ein breiteres Spektrum zu zeigen.
  \\
  Die erste MINT-Woche 2021 fand mit 6 Schüler statt. Im Jahr 2022 besuchten die NMS Fiss und Pians. 2023 wurde auch schon eine Woche verplant.
  Die Partnerfirmen stammen dabei meist aus dem Raum Landeck, werden mit der ersten MINT-Woche im Jahr 2023 aber auch mit der Fa. cookis in Imst erweitert.
  Die MINT-Woche wird im Allgemeinen gut angenommen und Berichte der Schulen bestätigen das Interesse.
  \subtopic{MINT-Lab Oberland}
  Präsentation und Vorstellung durch \obmann.\\

  \textbf{Leuchtturm Projekt MINT Lab Oberland} \\
  Zentraler Punkt in der Region, der aus den Tälern gut erreichbar ist. Perfekter Standort wurde am GYM Landeck gefunden, vor allem auch, da es mittels öffentlicher Verkehrsmittel gut erreichbar ist.
  Das Gymnasium Landeck stellt dem Verein die Räumlichkeiten, von 2 Räumen, kostenlos zur Verfügung.
  In Abstimmung mit der Bildungsdirektion wurden uns einige Bildungsstunden, zum Betrieb der Einrichtung zugesagt.\\
  \textbf{Vorstellung der Räume:}
  \begin{itemize}
    \item kreativ-Lab
    \item tech-Lab
    \item biochem-Lab
  \end{itemize}

  Die Schule kann über die Buchungsplattform das Lab buchen.
  Weiters können die Schulen aus den 3 Bereiche ein Workshop wählen und wechseln so in 3 Gruppen durch die Labs.
  Zudem besuchen auch zahlreiche Volksschulen der Umgebung das Lab, zB Workshop mit Beebots.
  Es folgt eine Präsentation verschiedener Impressionen aus dem Mint Lab.

  Der Obmann weiter:\\
  Nächster Planungsschritt sind kleinere Workshop-Kits zum Mitnehmen an die Schule. Manche Workshops sind so preisgünstig, dass man sie auch in die breite Masse an die Schulen ausrollen kann.
  Die Buchungsplattform ist seit 3 Wochen online und die Workshops finden jeden Mittwoch statt. Das MINT Lab ist jetzt schon bis 29.03.22 ausgebucht. \\\\ \textbf{Der Mint Branchentag der Bildungsdirektion} fand im im MINT Lab statt.\\

  \textbf{Förderungen für das MINT Lab}

  \begin{description}
    \item [Regio L LEADER-Programm:] Förderzusage des Landes Tirol über die Projektsumme von ca. 72.000€ mit einer Fördersumme von 60\%.
    \item [Sparkasse Imst Privatstiftung:] Für das nachhaltige Projekt MINT Lab erhielt der Verein die maximale Fördersumme von 10.000€.
  \end{description}

  Anzumerken ist: Der Verein stellt alle finanziellen Mittel zur Verfügung, bespielt werden muss das Lab aber vom Bildungssektor.

  \subtopic{Coding4Kids Herbstkurs}
  Projekt startet morgen 25.10.2022.\\
  Vorstellung Coding4Kids von Obmann \obmann.\\
  Neue Idee, die Kinder bei denen wir schon Interesse geweckt haben (\glqq MINI Nerds\grqq{}) m\"ussen wir am Ball halten. Herbstkurs findet in 8 Terminen teilweise online und teilweise im MINT Lab statt.
  Letzter Stand: Es gibt noch Restplätze. Bei maximal 12 Teilnehmer sind, Stand letzter Woche, 6 Anmeldungen.

  %---------------------------------------------------------

  \topic{Beschlussfassung über eingelangte Anträge}
  Keine Anträge eingelangt.

  \topic{Entgegennahme und Genehmigung des Rechenschaftsberichts und des Rechnungsabschlusses unter Einbindung der Rechnungsprüfer}
  Bericht von Rainer Haag:\\
  Kassabericht, mit Stichtag 14.10.2022. Die große Herausforderung: Viel Geld auftreiben. Das MINT Lab wurde mit einer Fördersumme von ca. 72.000€ und eine Förderquote von 60\% gefördert.
  Die Zahlung wurde jedoch noch nicht vom Land überwiesen.
  Eine Vorfinanzierung war notwendig, die der Verein über einige Leihgaben bekam.

  \textbf{Vorstellung Finanzbericht 2021}
  Mitgliedbeitrag: 380€, Leihgaben: 60.000€, Sponsorengelder: 49.994€.
  Überschuss 2021: 11.998€

  \textbf{Vorstellung Finanzbericht 2022}
  Mitgliedsbeitrag: 410€, Spenden: 705€, Sponsoren: 1.500€, Förderung: 10.000€, Offene Förderung: 43.036€\\

  Gesamtendüberschuss nach Rückzahlung der Leihgaben mit Einbezug der noch nicht bezahlten Förderung: 5.088€
  Leichter Überschuss, aber neue Ausgaben stehen schon vor der Tür.\\\\
  \textbf{Bericht der Kassaprüfer}

  Kassaprüfer Thomas Zangerl: \glqq Kassabericht hat gepasst\grqq{}

  \subtopic{Entlastung des Vorstands}
  \Onevote{Entlastung des Vorstandes. Abgestimmt durch Handzeichen, 16:35 }{9}{0}{0}

  %---------------------------------------------------------

  \topic{Allfälliges}
  ---
  \vspace{0.2\textheight}

  \signature{Obmann, \obmann}
  \hspace{0.2\columnwidth}
  \signature{Schriftführer, Simon Abler}
\end{Minutes}

% ----------------------------------------------------------------



\end{document}
% ----------------------------------------------------------------
