\documentclass{scrartcl}
\usepackage[utf8]{inputenc}
%\usepackage[ansinew]{inputenc}
\usepackage[german]{babel}
%\usepackage{german}
%\usepackage[TwoColumn]{minutes}
\usepackage{minutes}
\usepackage{caption}


% ----------------------------------------------------------------
\begin{document}

\begin{Minutes}{Protokoll der Jahreshauptversammlung vom 24.10.2022}
  \subtitle{Träger und Förerverein Mint Oberland}
  \moderation{Philipp Machac}
  \minutetaker{Simon Abler}
  \participant{Siehe Anwesenheitsliste
    \enclosure{Anwesenheitsliste}{Anwesenheitsliste}
  }

  \minutesdate{12th March 2014}
  \starttime{15:00}
  \endtime{17:00}
  \location{Landeck}
  %\missing{}
  \maketitle



  % Header end ---------------------------------------------------------





  \topic{Begrüßung und Feststellung der Beschlussfähigkeit}
  \topic{Wahl der Versammlungsleitung}

  \topic{Genehmigung des letzten Protokolls}

  \topic{Berichte des Vorstandes}
  \subtopic{Rückblick auf das letzte Jahr}
  \subtopic{Bericht des Vorstandes}
  \subtopic{Bericht des Schatzmeister/Kassenwart}
  \subtopic{Bericht des Kassenprüfers}

  \subtopic{Aussprache und Entlastung des Vorstandes}


  \Onevote{Entlastung des Vorstandes }{}{}{}

  \topic{Neuwahlen}
  \subtopic{Wahlverfahren}
  \begin{Vote}
    \vote{Abstimmung per Handzeichen}{}{}{}
    \vote{Blockwahl}{}{}{}
  \end{Vote}

  \subtopic{Wahl des Vorstandes}
  Zur Wahl als neuer Vorstand stehen:
  \begin{itemize}
    \item dfsdf
    \item sdfsdf
  \end{itemize}

  \begin{Vote}
    \vote{Abstimmung über den neuen Vorstand}{2}{3}{4}
    \vote{Annahme der Wahl durch den neuen Vorstand}{}{}{}
  \end{Vote}

  \begin{Argumentation}
    \Pro Ein Viertel der Kosten wird eingespart.
    \pro Der erhöhte Arbeitsaufwand bedeutet mehr gemeinsame Arbeit und stärkt das Zusammengehörigkeitsgefühl.
    \Contra Der Arbeitsaufwand wird erhöht ohne signifikante Verbesserungen zu bringen.
    \contra Der Arbeitsablauf verkompliziert sich.
    \result Die vorgeschlagene Änderung wird abgelehnt.
  \end{Argumentation}

  \subtopic{Wahl der Kassenprüfer}
  \begin{Vote}
    \vote{? als Kassenprüfer}{}{}{}
    \vote{Annahme der Wahl der Kassenprüferin}{}{}{}
  \end{Vote}



  \topic{Dates}
  \schedule{2000/12/24}{Christmas eve}
  \schedule{2000/12/24}[20:00]{distribution of presents}
  \schedule*{2000/12/25}{Christmas day (without entry in calendar)}

  %\subtopic{Wahl weiterer Ämter}

  %\topic{Satzungsänderungen}

  \topic{Verabschiedung des Haushaltsplans}
  \Onevote{Annahme des Haushaltsplanes}{}{}{}

  \topic{Anträge}

  \topic{Verschiedenes}

  \topic{Wer macht was bis wann? Verteilung der Aufgaben}
  \task{Petra Peterson}[15.08.2019]{Protokoll schreiben}


  \signature{Martin Muster}
  \signature{Petra Peterson}
\end{Minutes}

% ----------------------------------------------------------------
\newpage\thispagestyle{empty}
\newcounter{nummer}
\newcommand{\oneline}{
  \stepcounter{nummer}
  \arabic{nummer}\rule{0mm}{8mm}&&&&\\\hline
}

\begin{minipage}{\textwidth}
  \begin{tabular}{|r|p{5cm}|*{2}{|p{5mm}}|p{5cm}|}\hline
    \multicolumn{5}{|c|}{\bfseries Anwesenheitsliste}                    \\\hline\hline
    Nr. & Name, Vorname & \multicolumn{2}{l|}{Mitglied} & Unterschrift   \\
        &               & ja                            & nein         & \\\hline
    %% \loop in tabular klappt nicht
    %\oneline\oneline\oneline\oneline\oneline
    \oneline\oneline\oneline\oneline\oneline\oneline\oneline\oneline\oneline\oneline
    \oneline\oneline\oneline\oneline\oneline\oneline\oneline\oneline\oneline\oneline
  \end{tabular}
\end{minipage}


\end{document}
% ----------------------------------------------------------------
