%!TEX TS-program = xelatex
%!TEX encoding = UTF-8 Unicode

\documentclass{scrartcl}

\usepackage[utf8]{inputenc}
\usepackage{graphicx}

\usepackage[german]{babel}
\usepackage{minutes}
\usepackage{caption}

\usepackage[]{geometry}
\usepackage{fontspec}
\geometry{   top =50mm}
\setmainfont{DIN Next LT Pro Medium}
\usepackage{xcolor}


  \newcommand\vereinName{Träger und Förerverein Mint Oberland}
  \newcommand\address{Bruggfeldstraße 5, 6500 Landeck}
  \newcommand\phone{06221/29766}
  \newcommand\eMail{office@mint.tirol}




\definecolor{footercolor}{HTML}{595959}

\usepackage[]{scrlayer-scrpage}
\clearpairofpagestyles
\chead{\includegraphics{../../template/header.png}}
\ofoot{\pagemark}
\cfoot{\small\color{footercolor}
Träger- und Förderverein MINT Oberland | Bruggfeldstraße 5 | 6500 Landeck \par
office@mint.tirol | www.mint.tirol | ZVR-1223647736\par
AT09 3699 0000 0913 4099 | RBRTAT22
}


\minutesstyle{header={list},columns={1},contents={true}}


%\newpagestyle{myheader}{%
%\sethead{\includegraphics{../../template/header.png}}
%\headrule}


% ----------------------------------------------------------------

\begin{document}



\begin{Minutes}{Protokoll der Jahreshauptversammlung vom 24.10.2022}
  \subtitle{\vereinName}
  \moderation{\obmann}
  \minutetaker{Simon Abler}
  \participant{Siehe Anwesenheitsliste
    % \enclosure{Anwesenheitsliste}{Anwesenheitsliste}
  }

  \minutesdate{12th March 2014}
  \starttime{15:00}
  \endtime{17:00}
  \location{Lantech Besprechungsraum Klein}
  %\missing{}

  \maketitle


  % Header end ---------------------------------------------------------





  \topic{Begrüßung durch den Vorsitzenden des Vorstands}
  \subtopic{Feststellung der Beschlussfähigkeit}

  \topic{Ansprache des Obmannes}

  \topic{Projektvorstellungen}

  \subtopic{MINT-Woche 2021 und 2022}
  \subtopic{MINT-Lab Oberland}
  \subtopic{Coding4Kids Herbstkurs}


  \topic{Beschlussfassung über eingelangte Anträge}
  \topic{Entgegennahme und Genehmigung des Rechenschaftsberichts und des Rechnungsabschlusses unter Einbindung der Rechnungsprüfer}
  \subtopic{Entlastung des Vorstands}

  \topic{Allfälliges}


  \Onevote{Entlastung des Vorstandes }{}{}{}

  \topic{Neuwahlen}
  \subtopic{Wahlverfahren}
  \begin{Vote}
    \vote{Abstimmung per Handzeichen}{}{}{}
    \vote{Blockwahl}{}{}{}
  \end{Vote}

  \subtopic{Wahl des Vorstandes}
  Zur Wahl als neuer Vorstand stehen:
  \begin{itemize}
    \item dfsdf
    \item sdfsdf
  \end{itemize}

  \begin{Vote}
    \vote{Abstimmung über den neuen Vorstand}{2}{3}{4}
    \vote{Annahme der Wahl durch den neuen Vorstand}{}{}{}
  \end{Vote}

  \begin{Argumentation}
    \Pro Ein Viertel der Kosten wird eingespart.
    \pro Der erhöhte Arbeitsaufwand bedeutet mehr gemeinsame Arbeit und stärkt das Zusammengehörigkeitsgefühl.
    \Contra Der Arbeitsaufwand wird erhöht ohne signifikante Verbesserungen zu bringen.
    \contra Der Arbeitsablauf verkompliziert sich.
    \result Die vorgeschlagene Änderung wird abgelehnt.
  \end{Argumentation}

  \subtopic{Wahl der Kassenprüfer}
  \begin{Vote}
    \vote{? als Kassenprüfer}{}{}{}
    \vote{Annahme der Wahl der Kassenprüferin}{}{}{}
  \end{Vote}



  \topic{Dates}
  \schedule{2000/12/24}{Christmas eve}
  \schedule{2000/12/24}[20:00]{distribution of presents}
  \schedule*{2000/12/25}{Christmas day (without entry in calendar)}

  %\subtopic{Wahl weiterer Ämter}

  %\topic{Satzungsänderungen}

  \topic{Verabschiedung des Haushaltsplans}
  \Onevote{Annahme des Haushaltsplanes}{}{}{}

  \topic{Anträge}

  \topic{Verschiedenes}

  \topic{Wer macht was bis wann? Verteilung der Aufgaben}
  \task{Petra Peterson}[15.08.2019]{Protokoll schreiben}


  \signature{Martin Muster}
  \signature{Petra Peterson}
\end{Minutes}

% ----------------------------------------------------------------



\end{document}
% ----------------------------------------------------------------
